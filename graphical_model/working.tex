% An example graphical model, based on a general(ish) set of definitions
% Michael Lee (mdlee@uci.edu)
% Jan 2018
% 
% to convert to eps download GNU ps tools (https://sourceforge.net/projects/gnuwin32/?source=typ_redirect)
% then use (MSDOS) command line of form
% pdftops -eps graphicalModel.pdf test.eps
% to do second page, for example ...
% pdftops -eps -f 2 -l 2 graphicalModel.pdf test.eps

\documentclass[12pt]{article}

% packages
\usepackage{tikz}
\usepackage{array}
\usepackage{amsmath}
\usepackage{bm}
\usetikzlibrary{shadows,fit,positioning,arrows,intersections}

% stuff from somewhere
\makeatletter
\pgfkeys{/pgf/.cd,
  rectangle corner radius/.initial=3pt
}
\newif\ifpgf@rectanglewrc@donecorner@
\def\pgf@rectanglewithroundedcorners@docorner#1#2#3#4{%
  \edef\pgf@marshal{%
    \noexpand\pgfintersectionofpaths
      {%
        \noexpand\pgfpathmoveto{\noexpand\pgfpoint{\the\pgf@xa}{\the\pgf@ya}}%
        \noexpand\pgfpathlineto{\noexpand\pgfpoint{\the\pgf@x}{\the\pgf@y}}%
      }%
      {%
        \noexpand\pgfpathmoveto{\noexpand\pgfpointadd
          {\noexpand\pgfpoint{\the\pgf@xc}{\the\pgf@yc}}%
          {\noexpand\pgfpoint{#1}{#2}}}%
        \noexpand\pgfpatharc{#3}{#4}{\cornerradius}%
      }%
    }%
  \pgf@process{\pgf@marshal\pgfpointintersectionsolution{1}}%
  \pgf@process{\pgftransforminvert\pgfpointtransformed{}}%
  \pgf@rectanglewrc@donecorner@true
}
\pgfdeclareshape{rectangle with rounded corners}
{
  \inheritsavedanchors[from=rectangle] % this is nearly a rectangle
  \inheritanchor[from=rectangle]{north}
  \inheritanchor[from=rectangle]{north west}
  \inheritanchor[from=rectangle]{north east}
  \inheritanchor[from=rectangle]{center}
  \inheritanchor[from=rectangle]{west}
  \inheritanchor[from=rectangle]{east}
  \inheritanchor[from=rectangle]{mid}
  \inheritanchor[from=rectangle]{mid west}
  \inheritanchor[from=rectangle]{mid east}
  \inheritanchor[from=rectangle]{base}
  \inheritanchor[from=rectangle]{base west}
  \inheritanchor[from=rectangle]{base east}
  \inheritanchor[from=rectangle]{south}
  \inheritanchor[from=rectangle]{south west}
  \inheritanchor[from=rectangle]{south east}

  \savedmacro\cornerradius{%
    \edef\cornerradius{\pgfkeysvalueof{/pgf/rectangle corner radius}}%
  }

  \backgroundpath{%
    \northeast\advance\pgf@y-\cornerradius\relax
    \pgfpathmoveto{}%
    \pgfpatharc{0}{90}{\cornerradius}%
    \northeast\pgf@ya=\pgf@y\southwest\advance\pgf@x\cornerradius\relax\pgf@y=\pgf@ya
    \pgfpathlineto{}%
    \pgfpatharc{90}{180}{\cornerradius}%
    \southwest\advance\pgf@y\cornerradius\relax
    \pgfpathlineto{}%
    \pgfpatharc{180}{270}{\cornerradius}%
    \northeast\pgf@xa=\pgf@x\advance\pgf@xa-\cornerradius\southwest\pgf@x=\pgf@xa
    \pgfpathlineto{}%
    \pgfpatharc{270}{360}{\cornerradius}%
    \northeast\advance\pgf@y-\cornerradius\relax
    \pgfpathlineto{}%
  }

  \anchor{before north east}{\northeast\advance\pgf@y-\cornerradius}
  \anchor{after north east}{\northeast\advance\pgf@x-\cornerradius}
  \anchor{before north west}{\southwest\pgf@xa=\pgf@x\advance\pgf@xa\cornerradius
    \northeast\pgf@x=\pgf@xa}
  \anchor{after north west}{\northeast\pgf@ya=\pgf@y\advance\pgf@ya-\cornerradius
    \southwest\pgf@y=\pgf@ya}
  \anchor{before south west}{\southwest\advance\pgf@y\cornerradius}
  \anchor{after south west}{\southwest\advance\pgf@x\cornerradius}
  \anchor{before south east}{\northeast\pgf@xa=\pgf@x\advance\pgf@xa-\cornerradius
    \southwest\pgf@x=\pgf@xa}
  \anchor{after south east}{\southwest\pgf@ya=\pgf@y\advance\pgf@ya\cornerradius
    \northeast\pgf@y=\pgf@ya}

  \anchorborder{%
    \pgf@xb=\pgf@x% xb/yb is target
    \pgf@yb=\pgf@y%
    \southwest%
    \pgf@xa=\pgf@x% xa/ya is se
    \pgf@ya=\pgf@y%
    \northeast%
    \advance\pgf@x by-\pgf@xa%
    \advance\pgf@y by-\pgf@ya%
    \pgf@xc=.5\pgf@x% x/y is half width/height
    \pgf@yc=.5\pgf@y%
    \advance\pgf@xa by\pgf@xc% xa/ya becomes center
    \advance\pgf@ya by\pgf@yc%
    \edef\pgf@marshal{%
      \noexpand\pgfpointborderrectangle
      {\noexpand\pgfqpoint{\the\pgf@xb}{\the\pgf@yb}}
      {\noexpand\pgfqpoint{\the\pgf@xc}{\the\pgf@yc}}%
    }%
    \pgf@process{\pgf@marshal}%
    \advance\pgf@x by\pgf@xa% 
    \advance\pgf@y by\pgf@ya%
    \pgfextract@process\borderpoint{}%
    %
    \pgf@rectanglewrc@donecorner@false
    %
    % do southwest corner
    \southwest\pgf@xc=\pgf@x\pgf@yc=\pgf@y
    \advance\pgf@xc\cornerradius\relax\advance\pgf@yc\cornerradius\relax 
    \borderpoint
    \ifdim\pgf@x<\pgf@xc\relax\ifdim\pgf@y<\pgf@yc\relax
      \pgf@rectanglewithroundedcorners@docorner{-\cornerradius}{0pt}{180}{270}%
    \fi\fi
    %
    % do southeast corner
    \ifpgf@rectanglewrc@donecorner@\else
      \southwest\pgf@yc=\pgf@y\relax\northeast\pgf@xc=\pgf@x\relax
      \advance\pgf@xc-\cornerradius\relax\advance\pgf@yc\cornerradius\relax
      \borderpoint
      \ifdim\pgf@x>\pgf@xc\relax\ifdim\pgf@y<\pgf@yc\relax
       \pgf@rectanglewithroundedcorners@docorner{0pt}{-\cornerradius}{270}{360}%
      \fi\fi
    \fi
    %
    % do northeast corner
    \ifpgf@rectanglewrc@donecorner@\else
      \northeast\pgf@xc=\pgf@x\relax\pgf@yc=\pgf@y\relax
      \advance\pgf@xc-\cornerradius\relax\advance\pgf@yc-\cornerradius\relax
      \borderpoint
      \ifdim\pgf@x>\pgf@xc\relax\ifdim\pgf@y>\pgf@yc\relax
       \pgf@rectanglewithroundedcorners@docorner{\cornerradius}{0pt}{0}{90}%
      \fi\fi
    \fi
    %
    % do northwest corner
    \ifpgf@rectanglewrc@donecorner@\else
      \northeast\pgf@yc=\pgf@y\relax\southwest\pgf@xc=\pgf@x\relax
      \advance\pgf@xc\cornerradius\relax\advance\pgf@yc-\cornerradius\relax
      \borderpoint
      \ifdim\pgf@x<\pgf@xc\relax\ifdim\pgf@y>\pgf@yc\relax
       \pgf@rectanglewithroundedcorners@docorner{0pt}{\cornerradius}{90}{180}%
      \fi\fi
    \fi
  }
}

% Document starts here
\begin{document}

% use the whole page (especially for this example model, which needs it)
\setlength{\parindent}{-4cm}

% Colors for observed and partially observed nodes
\definecolor{lightgray}{RGB}{180,180,180}
\definecolor{verylightgray}{RGB}{220,220,220}

% Definitions of node types
\tikzstyle{discreteDeterministic} = [draw,  minimum size = 12mm, thick, draw = black, node distance = 10mm, fill  = white, line width = 1pt]
\tikzstyle{discreteObserved} = [draw,  minimum size = 10mm, thick, draw = black, node distance = 10mm, fill  = lightgray, line width = 1pt]
\tikzstyle{discreteLatent} = [draw,  minimum size = 10mm, thick, draw = black, node distance = 10mm, fill  = white, line width = 1pt]
\tikzstyle{continuousDeterministic} = [circle,  minimum size = 12mm, thick, draw = black, node distance = 10mm, fill  = white, line width = 1pt]
\tikzstyle{continuousObserved} = [circle, minimum size = 10mm, thick, draw = black, node distance = 10mm, fill = lightgray, line width = 1pt]
\tikzstyle{continuousLatent} = [circle, minimum size = 10mm, thick, draw = black, node distance = 10mm, line width = 1pt]
\tikzstyle{plate} = [rectangle with rounded corners, draw = black, rectangle corner radius = 6pt, align = center, line width = 1pt, yshift = -4]
\tikzstyle{spacer} = [opacity = 0, minimum size = 1mm] {};
\tikzstyle{connect} = [-stealth, thick, line width = 1pt]

% --------------------------------------------------------------------------------------------
% An example graphical model, using most of the defined features
% --------------------------------------------------------------------------------------------

% table with graphical model in left cell, and statistical definition in right cell
\begin{tabular}{ m{8.5cm} m{5.5cm}}

\begin{tikzpicture}[node distance = 2.5cm, auto , >=latex']
 
% Nodes
  \node [discreteLatent] (z) {$\bm z$};
  \node [discreteLatent] (tau) [left = 5mm of z] {$\bm\tau$};
  \node [discreteDeterministic] (wOuter) [below = 9mm of z] {};
  \node [discreteLatent] (w) [below = 10mm of z] {$w_i$};
  
  \node [continuousLatent] (ed) [right = 15mm of z] {$\epsilon^\mathrm{d}$};
  \node [continuousLatent] (es) [right = 1mm of ed] {$\epsilon^\mathrm{s}$};
  \node [continuousLatent] (er) [right = 1mm of es] {$\epsilon^\mathrm{r}$};
 
  \node [continuousDeterministic] (thetadOuter) [below = 9mm of w] {};
  \node [continuousLatent] (thetad) [below = 10mm of w] {$\theta_i^\mathrm{d}$};
 
  \node [continuousDeterministic] (thetasOuter) [right = 9mm of thetad] {};
  \node [continuousLatent] (thetas) [right = 10mm of thetad] {$\bm{\theta}_i^\mathrm{s}$};
 
  \node [continuousDeterministic] (thetarOuter) [right = 9mm of thetas] {};
  \node [continuousLatent] (thetar) [right = 10mm of thetas] {$\bm{\theta}_i^\mathrm{r}$};
 
 \node [discreteObserved] (s) [left = 10mm of thetad] {$\bm{s}_i$};
  
  \node [discreteObserved] (yd) [below = 10mm of thetad] {$y_i^\mathrm{d}$};
  \node [discreteObserved] (ys) [below = 10mm of thetas] {$\bm{y}_i^\mathrm{s}$};
  \node [discreteObserved] (yr) [below = 10mm of thetar] {$\bm{y}_i^\mathrm{r}$};
 
  % Graph structure
   \path (tau) edge [connect] (wOuter)
           (z) edge [connect] (wOuter)
           (wOuter) edge [connect] (thetadOuter)
           (wOuter) edge [connect] (thetasOuter)
           (wOuter) edge [connect] (thetarOuter)
           (ed) edge [connect] (thetadOuter)
           (es) edge [connect] (thetasOuter)
           (er) edge [connect] (thetarOuter)
           (s) edge [connect] (thetadOuter)
           (s) edge [connect, bend right] (thetasOuter)
           (s) edge [connect, bend right] (thetarOuter)
           (thetadOuter) edge [connect] (yd)
           (thetasOuter) edge [connect] (ys)
           (thetarOuter) edge [connect] (yr);
   
   % Plates
   \node[plate, inner sep = 5mm, fit = (yd) (s) (wOuter) (thetarOuter),  label={[anchor=south west]south west:$i$ trials}] {};
 
 \end{tikzpicture}
 
  % table break
  &
  
   % Statistical definition
   \renewcommand{\arraystretch}{1.2}
   $\begin{array}{rcl}
   \tau_k &\sim&  \mathrm{Categorical}\bigl(\overbrace{\frac{n}{2n-1},\frac{1}{2n-1}, \dots,\frac{1}{2n-1}}^{n\ \mathrm{trials}}); \tau_1 \leq \ldots \leq \tau_\gamma\\
   z_k & \sim & \mathrm{Categorical}\bigl(\frac{1}{4}, \frac{1}{4}, \frac{1}{4}, \frac{1}{4}\bigr)\\
   \epsilon^\mathrm{d}, \epsilon^\mathrm{s}, \epsilon^\mathrm{r} & \sim & \mathrm{Uniform}\bigl(0,\frac{1}{2}\bigr) \\
      w_i & = &z_x, \ \mathrm{where}\ x = \sum_k \mathcal{I}\left(i > \tau_k\right)\\
      \theta_i^\mathrm{d} & = & \left\{\begin{array}{ll}
   \mathcal{H}_{w_i}^\mathrm{d}\left(\bm{s}_i\right)\left(1-\epsilon^\mathrm{d}\right)+\frac{1}{2}\epsilon^\mathrm{d} & \mathrm{if}\  w_i\in\left\{\mathrm{ttb}, \mathrm{tally}, \mathrm{wadd}\right\}\\
   \frac{1}{2}& \mathrm{if}\  w_i=\mathrm{guess}\\
   \end{array}\right.\\
   \theta_{ik}^\mathrm{s} & = & \left\{\begin{array}{ll}
   \left(1-\epsilon^\mathrm{s}\right) & \mathrm{if}\  w_i\in\left\{\mathrm{ttb}, \mathrm{tally}, \mathrm{wadd}\right\},  \mathcal{H}_z^\mathrm{s}\left(\bm{s}_i\right)=1\\
   \epsilon^\mathrm{s} & \mathrm{if}\ w_i\in\left\{\mathrm{ttb}, \mathrm{tally}, \mathrm{wadd}\right\},  \mathcal{H}_z^\mathrm{s}\left(\bm{s}_i\right)=0\\
   \frac{1}{4} & \mathrm{if}\ w_i=\mathrm{guess}\\
   \end{array}\right.\\
   \theta_{ik}^\mathrm{r} & = & \left\{\begin{array}{ll}
   \left(1-\epsilon^\mathrm{r}\right) & \mathrm{if}\   \mathcal{H}_{w_i}^\mathrm{r}\left(\bm{s}_i\right)=1\\
   \epsilon^\mathrm{r} & \mathrm{if}\   \mathcal{H}_{w_i}^\mathrm{r}\left(\bm{s}_i\right)=0\\
   \frac{1}{2} & \mathrm{if}\   \mathcal{H}_{w_i}^\mathrm{r}\left(\bm{s}_i\right)=\frac{1}{2}\\
   \end{array}\right.\\
   y_i^\mathrm{d} & \sim & \mathrm{Bernoulli}\bigl(\theta_i^\mathrm{d}\bigr) \\
   y_{ik}^\mathrm{s} & \sim & \mathrm{Bernoulli}\bigl(\theta_{ik}^\mathrm{s}\bigr) \\
   y_{ik}^\mathrm{r} & \sim & \mathrm{Bernoulli}\bigl(\theta_{ik}^\mathrm{r}\bigr) \\
   \end{array}$
  
\end{tabular}

% Insert a \newpage then repeat for next graphical model in document
   
\end{document}

